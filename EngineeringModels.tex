\newpage
\section{Engineering Models}

Solidworks Product Data Management (PDM) Professional is the Telescope and Site group's official CAD model repository. It was selected during the design and development phase to meet the NSF requirement for configuration management of the baseline mechanical engineering design. The software uses a check-out / check-in system to allow configuration management of the design. Each check-in produces a new version of the part or assembly. Earlier versions can be accessed if needed to compare designs or revert to an earlier design. A workflow feature allows the designs to go through a review process until the design is approved and locked from further changes. A revision process is also included in the workflow to allow for changes to the designs if needed after final approval. The software allows for vault replication at multiple sites and the project currently has a server operating in Tucson and Chile to support CAD users at multiple sites.

The PDM system currently stores all the Telescope and Site group's CAD models of the original mechanical baseline design. The CAD models are linked to a series of ICD documents (drawings) describing the mechanical interfaces for the different optical and mechanical subsystems of the observatory. In addition to the baseline design models the PDM system contains CAD models of the as-designed and as-built CAD models from the subsystem vendors. Complete designs and documentation of subsystems completed by the Telescope and Site group are also included. In addition to CAD models a series of design and drafting standards is also stored in the PDM system.

The PDM system currently contains all design work created by the Telescope and Site group from early design and development through construction. No information has been deleted or archived at this time. The PDM system will be needed by the operations team as the observatory goes through its design lifecycle of revisions and upgrades. A reorganization of the PDM vault to archive obsolete information is envisioned at the end of the construction phase of the project.

{\bf Rubin PDM Vault Contents includes:}

\begin{itemize}
	\item {\bf Original Project Baseline Design Documentation}
	\begin{itemize}
		\item ICD drawings for original requirements documents
		\item CAD models of T\&S subsystems baseline design
		\item Library of CAD models contains commercial off the shelf parts
		\item Template files
		\begin{itemize}
			\item Project drawing sheets, part files and assemblies template files
			\item Project dimension standard files for Solidworks
			\item Project Tables (BML, Revision, hole tables, etc.)
		\end{itemize}
	\end{itemize}

	\item {\bf Complete designs (models \& drawings) for subsystems completed by the T\&S group}
	\begin{itemize}
		\item M1M3 mirror cell assembly
		\item M2 cone baffle
		\item Misc. handling fixtures
		\item ComCam Assembly
	\end{itemize}

	\item {\bf CAD models of T\&S subsystems from vendor contracts}
	\begin{itemize}
		\item Telescope Mount Assembly (FDR Model)
		\item M2 Cell Assembly
		\item M2 Hexapod Assembly
		\item Camera Hexapod \& Rotator Assembly
		\item Dome (FDR model)
		\item Coating Plant
		\item Aux Telescope \& Instrumentation
		\item Summit Facility includes Aux Telescope Facility
	\end{itemize}

	\item{\bf CAD model of SLAC Camera assembly} (last update July 2013)  Add description of the process for synchronizing the LSSTCam CAD model here

\end{itemize}
