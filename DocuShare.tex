\newpage
\section{DocuShare}

Xerox DocuShare content management system is the Construction project's official document repository. It was selected during the design and development phase to meet the NSF requirement for a document management system. During Construction, the NSF requires retention of, at minimum, record of any decision affecting the cost, schedule or performance baseline. The Major Facilities Guide specifies retention of a contingency log, change requests and approvals, and system integration, commissioning, testing and acceptance plans and results. During Operations, the NSF requires retention of documents related to facility performance in terms of maintenance, operating time, scheduled and unscheduled down time, and use for research and education. In addition to NSF requirements, the Project Office expects DocuShare to be the repository for official versions of management policies, plans and procedures, design documents, safety documentation, hazard analyses, released requirements and interface control documents generated from the SysML model, and project standards, guidelines and templates. The previous list is not intended to be exhaustive.

Three of DocuShare's advantages are handles, version control and co-location. Each object has a unique identifier called a handle, which follows the object regardless of versioning or location(s) in the directory structure. Each handle has a version history that lists all previously uploaded files for the object in question. One of those versions is designated as the preferred version, which represents the document's official, approved version and is served by the database when clicking the object's title or from a properly formatted URL shared or hosted outside of DocuShare, The object's handle does not change when/if a new version of the document is created. Lastly, objects can appear simultaneously in as many collections as are necessary and/or relevant to the document. This is facilitated by an object's "Locations" property; locations are added as appropriate, and the handle automatically appears in the newly added locations. The combination of handles, versioning and co-location creates a system where each document is represented by a single record, avoiding duplication and/or version confusion.

Currently, DocuShare contains more than 30,000 documents in more than 10,000 collections. A high-level survey of DocuShare shows retention of 1) requirements, designs, interfaces, policies and plans under change control - whether controlled at the project- or subsystem-level; 2) change control action records; 3) status reports to NSF, AURA and other stakeholders; 4) safety standards, accidents reports and investigations documentation; 5) hazard mitigation verification artifacts; 6) risk reports; and 7) project-level review agendas, presentations and reference documents. Creation, retention and version control of those documentation classes generally has been well managed; however, the bulk of DocuShare documents likely represents objects created ad hoc by general project staff for specific purposes. In addition, there are thousands of pieces of work product that may have archival significance but likely will not be useful for Operations.

A part of the Rubin Observatory Document Working group effort a study was conducted to evaluate the future use of DocuShare into Operations.  The full DocuShare Options Trade Study can be found here:  \url{https://docushare.lsst.org/docushare/dsweb/Get/Document-36788/DocuShareOptionsTradeStudy.pdf}.  this trade study includes examination of the following use cases:

\begin{itemize}
	\item Using the Existing Rubin Project DocuShare instance with modifications;
	\item Extending Rubin Project DocuShare with Archiver Server Enabled; and
	\item Adopting the NOIRLab DocuShare instance.
\end{itemize}

