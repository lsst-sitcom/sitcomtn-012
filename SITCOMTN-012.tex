\documentclass[SE,authoryear,toc]{lsstdoc}
% lsstdoc documentation: https://lsst-texmf.lsst.io/lsstdoc.html
\input{meta}

% Package imports go here.

% Local commands go here.

%If you want glossaries
%\input{aglossary.tex}
%\makeglossaries

\title{Rubin Construction Documentation Inventory}

% Optional subtitle
% \setDocSubtitle{A subtitle}

\author{%
Chuck Claver
}

\setDocRef{SITCOMTN-012}
\setDocUpstreamLocation{\url{https://github.com/lsst-sitcom/sitcomtn-012}}

\date{\vcsDate}

% Optional: name of the document's curator
% \setDocCurator{The Curator of this Document}

\setDocAbstract{%
This tech note bring together the sources of critical documentation that has been developed across the Rubin Observatory Construction Project.  This inventory is meant as a guide to the Rubin Operations team for how and what documentation is transferred  as a deliverable from the construction Project
}

% Change history defined here.
% Order: oldest first.
% Fields: VERSION, DATE, DESCRIPTION, OWNER NAME.
% See LPM-51 for version number policy.
\setDocChangeRecord{%
  \addtohist{1}{YYYY-MM-DD}{Unreleased.}{Chuck Claver}
}


\begin{document}

% Create the title page.
\maketitle
% Frequently for a technote we do not want a title page  uncomment this to remove the title page and changelog.
% use \mkshorttitle to remove the extra pages

% ADD CONTENT HERE
% You can also use the \input command to include several content files.

\appendix
% Include all the relevant bib files.
% https://lsst-texmf.lsst.io/lsstdoc.html#bibliographies
\section{References} \label{sec:bib}
\renewcommand{\refname}{} % Suppress default Bibliography section
\bibliography{local,lsst,lsst-dm,refs_ads,refs,books}

% Make sure lsst-texmf/bin/generateAcronyms.py is in your path
\section{Acronyms} \label{sec:acronyms}
\addtocounter{table}{-1}
\begin{longtable}{p{0.145\textwidth}p{0.8\textwidth}}\hline
\textbf{Acronym} & \textbf{Description}  \\\hline

AURA & Association of Universities for Research in Astronomy \\\hline
CCD & Charge-Coupled Device \\\hline
CCS & Camera Control System \\\hline
ComCam & The commissioning camera is a single-raft, 9-CCD camera that will be installed in LSST during commissioning, before the final camera is ready. \\\hline
DAQ & Data Acquisition System \\\hline
DM & Data Management \\\hline
DMTN & DM Technical Note \\\hline
DMTR & DM Test Report \\\hline
DOE & Department of Energy \\\hline
EFD & Engineering and Facility Database \\\hline
EPO & Education and Public Outreach \\\hline
EUI & Engineering User Interface System \\\hline
FDR & Final Design Review \\\hline
FRACAS & Failure Reporting Analysis and Corrective Action System \\\hline
HTML & HyperText Markup Language \\\hline
ICD & Interface Control Document \\\hline
IT & Information Technology \\\hline
LDM & LSST Data Management (Document Handle) \\\hline
LPM & LSST Project Management (Document Handle) \\\hline
LSE & LSST Systems Engineering (Document Handle) \\\hline
LSP & LSST Science Platform (now Rubin Science Platform) \\\hline
LSST & Legacy Survey of Space and Time (formerly Large Synoptic Survey Telescope) \\\hline
LVV & LSST Verification and Validation \\\hline
LaTeX & (Leslie) Lamport TeX (document markup language and document preparation system) \\\hline
M1M3 & Primary Mirror Tertiary Mirror \\\hline
M2 & Secondary Mirror \\\hline
MIE & Major Item of Equipment \\\hline
MREFC & Major Research Equipment and Facility Construction \\\hline
NSF & National Science Foundation \\\hline
PM & Project Manager \\\hline
PMO & Project Management Office \\\hline
PS & Project Scientist \\\hline
PST & Project Science Team \\\hline
PSTN & Project Science Technical Note \\\hline
RM & Release Manager \\\hline
RTN & Rubin Technical Note \\\hline
S3 & (Amazon) Simple Storage Service  \\\hline
SE & System Engineering \\\hline
SIT & System Integration, Test \\\hline
SLAC & SLAC National Accelerator Laboratory \\\hline
SQR & SQuARE document handle \\\hline
SQuaRE & Science Quality and Reliability Engineering \\\hline
SaaS & Software as a Service \\\hline
T\&S & Telescope and Site \\\hline
TMA & Telescope Mount Assembly \\\hline
TS & Test Specification \\\hline
URL & Universal Resource Locator \\\hline
\end{longtable}

% If you want glossary uncomment below -- comment out the two lines above
%\printglossaries





\end{document}
