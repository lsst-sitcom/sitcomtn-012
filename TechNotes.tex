\newpage
\section{TechNotes: https://www.lsst.io}

LSST the Docs (LTD), also known by its URL "lsst.io", is a documentation hosting platform built and operated by the SQuaRE (Data Management) team. LTD hosts what are called static websites, meaning any website built from HTML, CSS and JavaScript that doesn't need an active server to render content (as opposed to say Confluence, DocuShare, or Drupal websites). Static websites are a natural fit for documentation projects that originate from a GitHub repository, such as software documentation or LaTeX documents. Teams use the same processes to collaborate on documentation as they are with code, and the documentation is tested and built using the same process as the software is tested and built. LTD is unique in that is it built around versioned documentation. The root URL for a documentation project hosts the "default" version (which has a configurable meaning for each project). The user can also browse other versions of the documentation through the "/v/" dashboard pages (for example \url{https://www.lsst.io/v/}). Versions might correspond to software release versions or to temporary collaborative drafts corresponding to GitHub Pull Requests.

The homepage for the documentation platform, \url{https://www.lsst.io}, serves as a portal for documentation. \citep{lsst.io-cite} Users can search across metadata and full text (this feature is powered by the commercial service Algolia \citep{Algolia-cite} in conjunction with a scraper bot built by SQuaRE) or browse through curated collections. As \url{https://www.lsst.io} is a site designed and built by the SQuaRE team, there is considerable opportunity to refine the design of the portal to meet the specific needs of Rubin Observatory. It is also possible for the portal to list and provide search functionality for documentation hosted on other platforms (potential examples include Confluence, DocuShare, and the acronyms/terminology database).

From a technical perspective, LTD hosts two types of documentation projects: guides and documents. Guides are multi-page websites. The guideline in the DM subsystem is that every software project or service has a guide hosted on lsst.io. An example of a DM guide for a software project is \url{https://pipelines.lsst.io} and a guide for a service is \url{https://nb.lsst.io}. T\&S is also hosting software guides: see \url{https://obs-controls.lsst.io} as an example. Besides documentation tied to specific software projects or services, guides can also collect procedures for teams, see the DM Developer Guide (\url{https://developer.lsst.io}) or the Observatory Operations Documentation (\url{https://obs-ops.lsst.io}). Though not required, guides are generally authored using an open-source tool called Sphinx \citep{Sphinx-cite} using a theme that is maintained by the SQuaRE team. The second type of documentation, documents, are "single-page" artifacts that are analogous to documents that might be found in DocuShare. Indeed, DM has taken to developing most of its LDM change-controlled documents on its lsst.io site to take advantage of the sophisticated collaboration features that GitHub offers (for an example, see \url{https://ldm-151.lsst.io}). \citep{GitHub-cite} Change-controlled documents are submitted to DocuShare for archival once approved using a release process mediated through GitHub, Jira, and the relevant control board. LTD is currently hosting documents from the DMTR, LDM, LPM, LSE, and SCTR document series (note that this includes test and verification reports). Besides change-controlled documents, Rubin Observatory project members can also author informal documents called technical notes. Technical notes were introduced as a medium that blended the organization of DocuShare documents (documents have unique handles) with the ease of Confluence (staff can author and publish technical notes independently without assistance or overt oversight). 

Technical notes series are associated with different subsystems and include DMTN, ITTN, PSTN, RTN, SITCOMTN, SMTN, SQR, and TSTN. Counts of the number of available documents of each type are listed in the table below, as of June 2021:

\begin{longtable}{p{0.2\textwidth}p{0.2\textwidth}}\hline
\textbf{Content Type} & \textbf{Document Count}  \\\hline
Guides		& 59 		\\\hline
DMTN 		& 172	\\\hline
DMTR		& 22 		\\\hline
ITTN		& 34 		\\\hline
LDM			& 38 		\\\hline
LPM 		& 2 		\\\hline
PSTN  		& 51  	\\\hline
RTN   		& 12 		\\\hline
SCTR 		& 8 		\\\hline
SITCOMTN	& 12 		\\\hline
SMTN 		& 14		\\\hline
SQR 		& 49 		\\\hline
TSTN 		& 25          \\\hline          
\end{longtable}

\subsection{Terminology}

{\bf lsst.io:}  The documentation hosting domain for Rubin Observatory, which is powered by LSST the Docs. All subdomains of lsst.io are independent documentation projects (for example, developer.lsst.io for the DM Developer Guide or sqr-006.lsst.io for the SQR-006 technote).

{\bf \url{www.lsst.io}:}  The documentation portal. This is a regular documentation project hosted on lsst.io, but it uses the "www" subdomain to serve as a documentation homepage. \url{www.lsst.io} provides documentation search and faceted browsing capabilities. It is still in development and the current status is documented at \url{https://www.lsst.io/about/}. The site itself is built with React/Gatsby.js (\url{https://github.com/lsst-sqre/www_lsst_io}), the search database is SaaS \citep{SaaS-cite}), and the bot that indexes content into the search database is called Ook (\url{https://github.com/lsst-sqre/ook}).

{\bf LSST the DOCS (LTD):}  Commonly written as LTD, LSST the Docs is the technical system for hosting versioned static websites. lsst.io is one such deployment. The technical motivation and design of LTD are documented in SQR-006: The LSST the Docs Platform for Continuous Documentation Delivery (\url{https://sqr-006.lsst.io}, \citeds{SQR-006}). The key technical features of LTD are:

\begin{itemize}
	\item High reliability, scaling, and security: documentation is hosted in Amazon S3 and served through the Fastly content distribution network. We don't operate any servers that receive traffic from users;
	\item Versioned documentation; and
	\item Flexibility to host any type of static website.
\end{itemize}

{\bf Technical notes (technotes):}  These are documents that are not change controlled, but otherwise have features similar to "official" project documents. They were designed by the DM SQuaRE team as a replacement for Confluence in writing organized documents for sharing ideas within, and beyond, the project. See SQR-000: The LSST DM Technical Note Publishing Platform (https://sqr-000.lsst.io) for our motivation to create technical notes. \citedsp{SQR-000}
